\documentclass[a4paper,12pt,twoside]{article}
\usepackage{braket}
\usepackage{graphicx}
\usepackage{amsmath}
\usepackage[english]{babel}
\usepackage[utf8]{inputenc}
\usepackage[colorlinks,bookmarks=false,citecolor={darkgreen},linkcolor=blue,urlcolor=blue]{hyperref}
\usepackage{color}
\usepackage[T1]{fontenc}
\usepackage{float}
\usepackage{url}
\usepackage{lscape}
\usepackage{tikz}
\usetikzlibrary{patterns,decorations.pathreplacing}
\usepackage[stable]{footmisc}
\usepackage{wrapfig}
\usepackage{textcomp}
\usepackage{changepage}
\usepackage{booktabs}
\usepackage{subfig}
\usepackage{amssymb}
\usepackage[section]{placeins}
\usepackage{mathabx}
\usepackage{comment} 
\usepackage{listings}
\usepackage{verbatim}
\usepackage{multicol}



\paperheight=297mm
\paperwidth=210mm

\setlength{\textheight}{235mm}
\setlength{\topmargin}{-1.2cm} 
\setlength{\textwidth}{15cm}
\setlength{\oddsidemargin}{0.56cm}
\setlength{\evensidemargin}{0.56cm}

\pagestyle{plain}

% def
\def \be {\begin{equation}}
\def \ee {\end{equation}}
\def \dd  {{\rm d}}
\def \bf {\textbf}
\def \bi {\begin{itemize}}
\def \ei {\end{itemize}}
\def \ib {\item[$\bullet$]}
\def \H {{\mathcal H}}
\def \grad{\nabla}
\def \( {\left(}
\def \) {\right)}
\def \order {{\cal O}}
\def \bc {\begin{comment}}
\def \ec {\end{comment}}
\def \defr {\textcolor{red}{Definition:}}
\def \db {\textcolor{blue}}
\def \remv {\textcolor{darkgreen}{Remarques:}}
\def \met {\textcolor{violet}{Méthode }}
\def \prop {\textcolor{darkgreen}{Propriétés:}}
\def \R {$\mathbb{R}$}
\def \D {$\mathbb{D}$}


\definecolor{ciel}{rgb}{0.04,0.52,0.78} % bleu ciel
\definecolor{darkgreen}{rgb}{0,0.5,0}% vert foncé
\definecolor{marron}{rgb}{0.32,0.19,0.19}%marron
\definecolor{violet}{rgb}{0.48,0.06,0.89}% violet
\definecolor{jaune}{rgb}{0.9,0.7,0.0} %jaune fonce


\def \swag{\textcolor{red}{S}\textcolor{ciel}{W}\textcolor{jaune}{A}\textcolor{darkgreen}{G}}



\newcommand{\mail}[1]{{\href{mailto:#1}{#1}}}
\newcommand{\ftplink}[1]{{\href{ftp://#1}{#1}}}
\newcommand{\e}{{\mathrm e}}
\renewcommand{\labelitemii}{$-$}


\newcommand{\thetaxy}{\theta_{X|Y }}
\newcommand{\thetayx}{\theta_{Y|X}}

\newcommand{\Nxy}{{N_{X|Y}}}
\newcommand{\Nyx}{{N_{Y|X}}}
\newcommand{\Nyy}{{N_{Y|Y}}}
\newcommand{\Nxx}{{N_{X|X}}}



\newcommand{\dxy}{{\delta_{X|Y}}}
\newcommand{\dyx}{{\delta_{Y|X}}}
\newcommand{\dyy}{{\delta_{Y|Y}}}
\newcommand{\dxx}{{\delta_{X|X}}}
\newcommand{\X}{\textcolor{red}{X}}
\newcommand{\Y}{\textcolor{blue}{Y}}

\title{Eigenfunction expansion of the refractory density} 

\author{No\'e Gallice \\ Professor: Wulfram Gerstner \hspace{0.5cm}  Supervisor: Tilo Schwalger\\ \small{ Laboratory of Computational Neuroscience, EPFL }
}

\date{January 8, 2018}


% ======= Le document commence ici ======

\begin{document}

\section{Emission rate equation}

%As the density $p(v,t)$ with the Fokker-Planck equation, the refractory density $q(\tau,t)$ can be expressed with the eigenfunctions  $ \{\ket{\phi_n}\}$. 

\begin{equation}
\label{eq:q}
\ket{q}=\sum_na_n\ket{\phi_n}
\end{equation}

where $a_n=\langle \psi_n | q\rangle$ are the time dependent coefficients of the modal expansion. In particular from Eq.\eqref{eq:psi0}, and Eq.\eqref{eq:dij} we have $a_0(t)=1$. The dynamics of the $a_n$ can be determined directly using Eq.\eqref{eq:Loperator}, and Eq.\eqref{eq:masterequation2}

\begin{align}
\dot{a}_n&=\langle\psi_n|\partial_t q\rangle+\langle\partial_t\psi_n|q\rangle \nonumber \\
&=\langle\psi_n|\mathcal{L}q\rangle+  \dot{h}\sum_ma_m\langle\partial_h\psi_n|\phi_m \rangle \nonumber \\
&=\lambda_n a_n +  \dot{h}\sum_ma_m\langle\partial_h\psi_n|\phi_m \rangle 
\end{align}

Defining the coupling coefficient as $C_{nm}=\langle\partial_h\psi_n|\phi_m \rangle $ we can rewrite

\begin{equation}
\dot{a}_n=\lambda_n a_n +  \dot{h}\sum_mC_{nm}a_m 
\end{equation}



We can finally express the activity $A(t)=q(0,t)$ as

\begin{equation}
\label{eq:A2}
A(t)=\sum_na_n(t)\phi_n(0)
\end{equation}

Keeping only the first mode, and using the fact that $\ket{\phi_{-n}}=\ket{\bar{\phi}_n}$ and $a_{-n}=\bar{a}_n$,  Eq.\eqref{eq:A2} becomes

\begin{align}
\label{eq:A3}
A(t)&=\phi_0(0) + a_1\phi_1(0) +a_{-1}\phi_{-1}(0) \nonumber \\
&=\phi_0(0) + 2\left(\Re\big[a_1\big]\Re\big[\phi_1(0)\big]- \Im\big[a_1\big]\Im\big[\phi_1(0)\big]\right)
\end{align}


And the dynamics of the $a_1$ is given by
\begin{equation}
\label{eq:a1}
\dot{a}_1=\lambda_1 a_1 + \dot{h}\big[C_{10}+C_{11}a_1+C_{1-1}a_{-1}\big] \nonumber \\
\end{equation}

Separating explicitly the real part $X(t)$ and the imaginary part  $Y(t)$ of $a_1(t)$

\begin{equation}
\label{eq:a1xy}
a_1(t)= X(t) +i Y(t)
\end{equation}

we derived from Eq.\eqref{eq:a1}  two non linear differential equation

\begin{align}
\dot{X}=&\Re[f]X-\Im[g]Y +\Re[C_{10}]\dot{h}\\
\dot{Y}=&\Re[g]Y+\Im[f]X +\Im[C_{10}]\dot{h}\\
\end{align}

with
\begin{align}
f=&\lambda_1+ \dot{h}(c_{11}+c_{1-1})\\
g=&\lambda_1+ \dot{h}(c_{11}-c_{1-1})
\end{align}

We have finally a set of three non linear differential equation $\dot{h}$, $\dot{X}$, $\dot{Y}$



\end{document} %%%% THE END %%%%