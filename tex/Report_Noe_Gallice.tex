\documentclass[a4paper,12pt,twoside]{article}

\usepackage{graphicx}
\usepackage{amsmath}
\usepackage[english]{babel}
\usepackage[utf8]{inputenc}
\usepackage[colorlinks,bookmarks=false,citecolor={darkgreen},linkcolor=blue,urlcolor=blue]{hyperref}
\usepackage{color}
\usepackage[T1]{fontenc}
\usepackage{float}
\usepackage{url}
\usepackage{lscape}
\usepackage{tikz}
\usetikzlibrary{patterns,decorations.pathreplacing}
\usepackage[stable]{footmisc}
\usepackage{wrapfig}
\usepackage{textcomp}
\usepackage{changepage}
\usepackage{booktabs}
\usepackage{subfig}
\usepackage{amssymb}
\usepackage[section]{placeins}
\usepackage{mathabx}
\usepackage{comment} 
\usepackage{listings}
\usepackage{verbatim}
\usepackage{multicol}



\paperheight=297mm
\paperwidth=210mm

\setlength{\textheight}{235mm}
\setlength{\topmargin}{-1.2cm} 
\setlength{\textwidth}{15cm}
\setlength{\oddsidemargin}{0.56cm}
\setlength{\evensidemargin}{0.56cm}

\pagestyle{plain}

% def
\def \be {\begin{equation}}
\def \ee {\end{equation}}
\def \dd  {{\rm d}}
\def \bf {\textbf}
\def \bi {\begin{itemize}}
\def \ei {\end{itemize}}
\def \ib {\item[$\bullet$]}
\def \H {{\mathcal H}}
\def \grad{\nabla}
\def \( {\left(}
\def \) {\right)}
\def \order {{\cal O}}
\def \bc {\begin{comment}}
\def \ec {\end{comment}}
\def \defr {\textcolor{red}{Definition:}}
\def \db {\textcolor{blue}}
\def \remv {\textcolor{darkgreen}{Remarques:}}
\def \met {\textcolor{violet}{Méthode }}
\def \prop {\textcolor{darkgreen}{Propriétés:}}
\def \R {$\mathbb{R}$}
\def \D {$\mathbb{D}$}


\definecolor{ciel}{rgb}{0.04,0.52,0.78} % bleu ciel
\definecolor{darkgreen}{rgb}{0,0.5,0}% vert foncé
\definecolor{marron}{rgb}{0.32,0.19,0.19}%marron
\definecolor{violet}{rgb}{0.48,0.06,0.89}% violet
\definecolor{jaune}{rgb}{0.9,0.7,0.0} %jaune fonce


\def \swag{\textcolor{red}{S}\textcolor{ciel}{W}\textcolor{jaune}{A}\textcolor{darkgreen}{G}}



\newcommand{\mail}[1]{{\href{mailto:#1}{#1}}}
\newcommand{\ftplink}[1]{{\href{ftp://#1}{#1}}}
\newcommand{\e}{{\mathrm e}}
\renewcommand{\labelitemii}{$-$}


\newcommand{\thetaxy}{\theta_{X|Y }}
\newcommand{\thetayx}{\theta_{Y|X}}

\newcommand{\Nxy}{{N_{X|Y}}}
\newcommand{\Nyx}{{N_{Y|X}}}
\newcommand{\Nyy}{{N_{Y|Y}}}
\newcommand{\Nxx}{{N_{X|X}}}



\newcommand{\dxy}{{\delta_{X|Y}}}
\newcommand{\dyx}{{\delta_{Y|X}}}
\newcommand{\dyy}{{\delta_{Y|Y}}}
\newcommand{\dxx}{{\delta_{X|X}}}
\newcommand{\X}{\textcolor{red}{X}}
\newcommand{\Y}{\textcolor{blue}{Y}}

\title{Eigenfunction expansion of the refractory density} 

\author{No\'e Gallice \\ Professor: Wulfram Gerstner \hspace{0.5cm}  Supervisor: Tilo Schwalger\\ \small{ Laboratory of Computational Neuroscience, EPFL }
}

\date{January 8, 2018}


% ======= Le document commence ici ======

\begin{document}

\maketitle

\tableofcontents % Table des matieres

\baselineskip=16pt
\parindent=15pt
\parskip=5pt


%%%%%%%%%%%%%%%%
%
%  ---- DISCUSSION------
%
%%%%%%%%%%%%%%%%
\newpage

Master equation

\be
\label{masterequation}
\frac{\partial q}{\partial t}=-\frac{\partial q}{\partial \tau}-\rho(\tau)q
\ee

boundary condition

\begin{align}
\label{boundarycondition}
q(0,t)=\int_{0}^{\infty}\rho(\tau)q(\tau,t)d\tau=A(t) \\
q(\infty,t)=0
\end{align}

$q$ is normalised 

\be
\label{nomalisation}
q(0,t)=\int_{0}^{\infty}q(\tau,t)d\tau \\
\ee


We can expand the refractory density
\be
\label{refractoryexpansion}
q(\tau,t)=\sum_n a_n(t)\phi_n(\tau)
\ee

where $\phi_n(\tau)$ are the eigenfunctions of the operator $\mathcal{L}=-\partial_{\tau}-\rho(\tau)$


\be
\label{Loperator}
\mathcal{L}\phi_n=\lambda_n\phi_n
\ee

if the eigenvalues $\lambda_n$ are complex, the complex conjugate of an eigenvalue is also an eigenvalue beacause $\mathcal{L}$ is a real operator

Because $\mathcal{L}$ cannot be generally brought to an Hermitian form we also need  the eigenfunction $\psi_n$ of the ajoint operator  $\mathcal{L}^{+}$

\be
\label{Ldegaoperator}
\mathcal{L}^+\psi_n=\lambda_n^+\psi_n
\ee

Defining the inner product one can show that the eigenvalues of eq.\eqref{Loperator} and eq.\eqref{Ldegaoperator} are the same:

\be
(\psi,\phi)=\int_{0}^{\infty}\psi(\tau)\phi(\tau)d\tau
\ee

\begin{align}
\lambda_n(\psi_n,\phi_n) &=\int_{0}^{\infty}\psi(\tau)\mathcal{L}\phi(\tau)d\tau  \nonumber \\
&=(\psi_n,\mathcal{L}\phi_n)  \nonumber \\
&=(\mathcal{L}^+\psi_n,\phi_n)  \nonumber \\
&=\int_{0}^{\infty}\mathcal{L}^+\psi_n(\tau)\phi_n(\tau)d\tau  \nonumber \\
&=\lambda_n^+(\psi_n,\phi_n) \label{lndega}
\end{align}

Eq.\eqref{lndega} implies that $\lambda_n=\lambda_n^+$ and
	
\be
\label{Ldegaoperator2}
\mathcal{L}^+\psi_n=\lambda_n\psi_n
\ee

For different eigenvalues, the eigenfunctions $\psi_i$ and $phi_j$ are orthogonal:

\begin{align}
\lambda_j(\psi_i,\phi_j) 
&=(\psi_i,\mathcal{L}\phi_j) \nonumber \\
&=(\mathcal{L}^+\psi_i,\phi_j)  \nonumber \\
&=\lambda_i(\psi_i,\phi_j) \label{lorthogonal}
\end{align}

We may thus normalize the functions according to 

\be
\label{dij}
(\psi_i,\phi_j)=\delta_{ij}
\ee

If a stationary solution of Matser equation exists we have:

\be
\lambda_0=0\:, \hspace{0.7cm} \phi_0(\tau)=q_{st}(\tau)\:,\hspace{0.7cm} \psi_0(\tau)=1
\ee

We can find the adjoint operator $\mathcal{L}$, using the integration by part:

\begin{align}
(\psi,\mathcal{L}\phi)&= \int_{0}^{\infty}\psi(\tau)\mathcal{L}\phi(\tau)d\tau  \nonumber \\
&= \int_{0}^{\infty}\psi(\tau)[-\partial_{\tau}-\rho(\tau)]\phi(\tau)d\tau  \nonumber \\
&=-[\psi(\tau)\phi(\tau)]^{\infty}_{0}+\int_{0}^{\infty}\partial_{\tau}\psi(\tau)\phi(\tau)d\tau -\int_{0}^{\infty}\rho(\tau)\psi(\tau)\phi(\tau)d\tau \nonumber \\
&= \psi(0)\phi(0)+ \int_{0}^{\infty}[\partial_{\tau}-\rho(\tau)]\psi(\tau)\phi(\tau)d\tau  \nonumber \\
&=\int_{0}^{\infty} \psi(0)\rho(\tau)\phi(\tau)d\tau+ \int_{0}^{\infty}[\partial_{\tau}-\rho(\tau)]\psi(\tau)\phi(\tau)d\tau  \nonumber \\
&= \int_{0}^{\infty}\{[\partial_{\tau}-\rho(\tau)]\psi(\tau)+ \psi(0)\rho(\tau)\}\phi(\tau)d\tau  \nonumber \\
& = (\mathcal{L}^+\psi,\phi)
\end{align}

with 
\be
\label{Ldega}
\mathcal{L}^+\psi(\tau)=[\partial_{\tau}-\rho(\tau)]\psi(\tau)+\psi(0)\rho(\tau)
\ee

From eq.\eqref{dij} and eq.\eqref{refractoryexpansion} we deduce that:

\be
\label{an}
a_n=(\psi_n,q)
\ee

Taking the derivative of $a_n$ with respect to time we have:

\begin{align}
\label{anderiv}
\frac{d a_n}{dt}&=(\psi_n,\partial_tq) \nonumber \\
&=(\psi_n,\mathcal{L}q)  \nonumber \\
&=(\mathcal{L}^+\psi_n,q) \nonumber \\
&=\lambda_n(\psi_n,q) \nonumber \\
&=\lambda_na_n
\end{align}

The solution of eq.\eqref{anderiv} with initial refractory density $q(0,\tau)$ is:

\begin{align}
\label{ant}
a_n(t) = &a_n(0)\exp(\lambda_nt)\\
with \hspace{0.4cm} a_n(0) = & \int_{0}^{\infty}\psi_n(\tau)q(0,\tau)d\tau
\end{align}

The solution eq.\eqref{Loperator}  and eq.\eqref{Ldegaoperator2} with initial refractory density $q(0,\tau)$ is:

\begin{align}
\label{phin}
\phi_n(\tau)&=\phi_n(0)\exp(-\lambda_n\tau-\int_0^\tau\rho(s)ds)\nonumber\\
				  &=\phi_n(0)\exp(-\lambda_n\tau)S(\tau)
\end{align}

\begin{align}
\label{psin}
\psi_n(\tau)&=\psi_n(0)\exp\big(\lambda_n\tau+\int_0^\tau\rho(s)ds\big)\nonumber\big[1-\int^\tau_0 \rho(x) \exp\big(-\lambda_nx-\int_0^\tau\rho(s)ds\big)dx\big]\\
&=\psi_n(0)\exp(\lambda_n\tau)S^{-1}(\tau)\big[1-\int^\tau_0 P(x) \exp\big(-\lambda_nx)dx\big]
\end{align}



Inserting eq.\eqref{phin} et eq.\eqref{psin} in eq.\ref{dij} we have:

\begin{align}
1=&\int_0^{\infty}\phi_n(0)\psi_n(0)\big[1-\int^\tau_0 P(x) \exp\big(-\lambda_nx)dx\big]d\tau \\
\phi_n(0)\psi_n(0) =&\frac{1}{\int_0^{\infty}\big[1-\int^\tau_0 P(x) \exp\big(-\lambda_nx)dx\big]d\tau}
\end{align}

In particular for $n=0$, $\lambda_0=0$ and $\psi_0(0)=1$, so we recover the relation:

\be
\phi_0(0) = \frac{1}{\int_0^{\infty}S(\tau)d\tau}
\ee

Inserting eq.\eqref{phin} for $n=0$ in eq.\ref{dij} we have:

\begin{align}
\int_0^{\infty}\phi_0(0)S(\tau) =& 1 \\
\phi_0(0) =& \frac{1}{\int_0^{\infty}S(\tau)}
\end{align}

Inserting eq. \eqref{ant} in eq.\eqref{todo}


For a LIF neuron with exponential link function the hazard rate is given by:

$\rho(\tau,t)=C\exp(\frac{u(\tau,t)-V_{th}}{\Delta})$

with membrane potential:
$u(\tau,t)=V_r e^{-\tau/\tau_m}+\frac{1}{\tau_m}\int_0^\tau  e^{-s/\tau_m}\mu(t-s))ds$

$\tau_m\frac{du(\tau,t)}{d\tau}=-u(\tau,t)+\mu(t)$

$V_{th}=15  mV$
$\Delta = 2     mV$
$C=1000 Hz$
$dt=0.1    ms$
$tau=20   ms$
$V_r=0  $


In theoretical neuroscience one common way of understanding neuronal dynamics in the brain is to pass from complex system to simplified model and the last step is to derived mathematical tractable equation, that we can well undertsand and where we can apply  tools from dynamical system theory for example. 

For population density equation the last step is not trivial. and we dont have a simple ODE for the firing rate.

My work is focus on this step and based on the work where they derived...


\newpage

\mbox{}
\nocite{*}
%\bibliographystyle{plain}
\bibliographystyle{unsrt}
\bibliography{biblio}

\end{document} %%%% THE END %%%%